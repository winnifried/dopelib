\subsubsection{General problem description}
This example implements the minimum compliance problem for the thickness optimization 
of an MBB-Beam. Using the MMA-Method of K. Svanberg together with an augmented Lagrangian 
approach for the subproblems following M. Stingl.
%\todo{Gleichung, reference}

The implementation is done using the following three additional files:

\begin{itemize}
\item \texttt{generalized\underline{ }mma\underline{ }algorithm.h}
  An implementation of the MMA-Algorithm for structural optimization using an augmented
  Lagrangian formulation for the subproblems. The subproblem is implemented using the 
  special purpose\\ file \texttt{augmentedlagrangianproblem.h}.
\item \texttt{augmentedlagrangianproblem.h} The problem container 
  for the augmented Lagrangian problem.
\item \texttt{voidreducedproblem.h} A wrapper file that eliminates $u$ 
  if it is not present anyways. This is used so that we can use the same routines to 
  solve problems that have no PDE constraint. This is used to fit the augmentedlagrangian 
  problem into our framework.
\end{itemize}


\subsubsection{Program description}
\textcolor{red}{
In addition to the previous Example \ref{OPT_Stat_MBB-Beam}, we consider now 
in addition 1 global constraint, which is realized by setting $1$ in the 
second argument of \texttt{constraints(lcc, 1)}. 
We use 
\texttt{localconstraints.h} and \texttt{localconstraintaccessor.h} to impose 
all constraints. First, we have again one control block with a lower and 
an upper bound, 
\[
\rho_{min} \leq q \leq \rho_{max}
\]
with $\rho_{min} = 10^{-4}$ and $\rho_{max} = 1$ ($\rho$ denotes the density
of the material). These are implemented in \texttt{localconstraintaccessor.h}.
The global constraint is the maximum volume of the material,
which should remain constant with the value $0.54$. Its implementation is 
provided in \texttt{localconstraints.h}.
}
