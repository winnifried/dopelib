\subsubsection{General problem description}
The modeling part of this example is based on the coupled
Biot-Lam\'e-Navier system. The Biot system itself   
is a standard model in subsurface modeling \cite{Biot41a, Biot41b, Biot55}. 
Here, a reservoir (the
pay-zone) is modeled as a poroelastic medium with the help of Biot's
equations. A surrounding medium (the non-pay zone) is modeled as a static
elastic solid. 
In fact, the
Biot system is a multi-scale problem which 
is identified on the micro-scale as 
a fluid-structure interaction problem (details on the 
interface law are found in Mikeli\'c and Wheeler (2011). 
This system 
is specifically suited for applications 
in subsurface modeling for the poroelastic part, the 
so-called pay-zone.  
On the other hand, surrounding rock 
(the non-pay zone) 
is modeled with the help of linear elasticity (Ciarlet 1984). 
Therefore, the final configuration 
belongs to a multiphysics problem in non-overlapping domains.
The nonstationary coupled system for the state is formulated 
within a variational monolithically-coupled framework, 
which is known 
to be more robust than partitioned solutions algorithms.
Its discretization 
is carried out with help of the Rothe method in which we first 
discretize in time and then in space. 
The configuration 
is based on the augmented Mandel problem which 
shows the important Mandel-Cryer effect: First increasing pressure
and then decreasing pressure in time while applying some 
traction force on the top boundary. 

We begin by describing the setting for a pure poroelastic setting, 
the so-called pay-zone.
Let $\Omega_B$ the domain of interest and $\partial\Omega_B$ its boundary
with the partition:
\begin{equation*}
\partial\Omega_B = \Gamma_u \cup \Gamma_t = \Gamma_p \cup \Gamma_f,
\end{equation*}
where $\Gamma_u$ denotes the displacement boundary (Dirichlet), $\Gamma_t$ the 
total stress or traction boundary (Neumann), $\Gamma_p$ the pore 
pressure boundary (Dirichlet), and $\Gamma_f$ the fluid flux boundary
(Neumann). Concretely, we have for a material 
with the displacement variable $u$ and its 
Cauchy stress tensor $\sigma$:
\begin{align*}
u &= \bar{u} \quad \text{on } \Gamma_u, \\
\sigma n &= \bar{t} \quad \text{on } \Gamma_t, 
\end{align*}
for given $\bar{u}$ and $\bar{t}$, and the 
normal vector $n$. 
For the pressure with the permeability tensor 
$K$ and fluid's viscosity $\eta_f$, we have the conditions:
\begin{align*}
p &= \bar{p} \quad \text{on } \Gamma_p, \\
- \frac{K}{\eta_f} \Bigl( \nabla p - \rho_f g\Bigr) \cdot n &= \bar{q}\quad \text{on } \Gamma_f,
\end{align*}
for given $\bar{p}$ and $\bar{q}$; and the density $\rho_f$ and the gravity $g$.
For the initial conditions at time $\tau=0$, we prescribe
\begin{align*}
p(\tau =0) = p_0 \quad \text{in } \Omega_B,\\
\sigma(\tau =0) = \sigma_0 \quad \text{in } \Omega_B,
\end{align*}
In this case of this extension
(if $\Omega_B$ is totally embedded in $\Omega_S$)
the boundary conditions on $\partial\Omega_B$ reduce 
to interface conditions $\partial\Omega_B :=\Gamma_i =
\Omega_B \cap \Omega_S$.
Let $I:=[0,T]$ denote the time interval.

\begin{Problem}[The Biot system]
Find the pressure $p_{\text{B}}$ and displacement 
$u_{\text{B}}$ such that
\begin{align*}
\partial_t (c_B p_{\text{B}} + \alpha_B \text{div}u_{\text{B}})\\ -
\frac{1}{\eta_f} \text{div}\, K 
(\nabla p_{\text{B}} - \rho_f g) &= q \quad \text{in }\Omega_{B} \times I, \\
- \text{div}\bigl( \sigma_B (u) \bigr)
+ \alpha_B\nabla p_{\text{B}} &= f_{\text{B}} 
\quad \text{in }\Omega_{\text{B}} \times I,
\end{align*}
with
\[
\sigma_B(u_B) := \mu_B(\nabla u_{\text{B}} + \nabla u_{\text{B}}^T) +  
\lambda_B\text{div}\, u_{\text{B}}I,
\]
and the coefficients $c_B\geq 0$, 
the Biot-Willis constant $\alpha_B \in [0,1]$,
(in fact, this constant relates to the amount of 
coupling between the flow part and the elastic part) 
and the permeability tensor $K$, 
fluid's viscosity and its density 
$\eta_f$ and $\rho_f$, gravity $g$ and a volume 
source term $q$ (i.e., usually, wells for oil production and
fluid injection). In the second equation,
the Lam\'e coefficients are denoted 
by $\lambda_B >0$ and $\mu_B >0$ 
and $f_{\text{B}}$ is a volume force.
\end{Problem}

The velocity $v_{\text{B}}$ 
in the porous medium is obtained 
with the help of Darcy's law (Darcy 1856)
and the Darcy equations which are obtained 
through homogenization of Stokes's equations
It holds:
\begin{equation*}
v_{\text{B}} = -\frac{1}{\eta_f} K(\nabla p_{\text{B}} - \rho_f g).
\end{equation*}


Usually the non-pay zone is described 
in terms of linear elasticity: 
\begin{Problem}
Find 
a displacement $u_S$ such that
\begin{align*}
- \text{div}\bigl( \sigma_S (u_S)  \bigr) = f_S \quad \text{in }\Omega_{\text{S}} \times I,
\end{align*}
with
\begin{align*}
\sigma_S (u_S) := \mu_S(\nabla u_{\text{S}} + \nabla u_{\text{S}}^T) +  
\lambda_S\text{div}\, u_{\text{S}}I, 
\end{align*}
with the Lam\'e coefficients $\mu_S$ and $\lambda_S$
and a volume force $f_S$. On the boundary 
$\partial\Omega_S := \Gamma_D \cup \Gamma_N$, the 
conditions
\begin{equation*}
u_S = \bar{u}_S \quad\text{on } \Gamma_D , \quad \sigma_S (u_S) n_S = \bar{t}_S \quad\text{on } \Gamma_N,
\end{equation*}
are prescribed with given $\bar{u}_S$ and $\bar{t}_S$.
\end{Problem}

It finally remains to describe the interface 
conditions on $\Gamma_{i}$ between the two sub-systems:
\begin{equation}
\label{interface_conditions}
\begin{aligned}
u_{\text{B}} &= u_S ,   \\
\sigma_{B}(u_B)n_B - \sigma_S(u_S)n_S &= 
\alpha p_{\text{B}} n_B  , \\
-\frac{1}{\eta_f} K(\nabla p_{\text{B}} - \rho_f g) \cdot n_S &= 0. 
\end{aligned} 
\end{equation}


\subsubsection{Program description}
In this example, the crucial aspect (from mathematical point of view 
as well as from the implementation) are the interface conditions
\eqref{interface_conditions}.
Here, it is important to notice that the second condition 
in \eqref{interface_conditions}, requires careful 
implementation on the interface, which is 
carried out with the help of deal.II's FE Nothing element.
Second, please do not forget to activate the flag
\begin{verbatim}
      HasFaces() const
      {
        return true;
      }
\end{verbatim}

The problem is driven by traction forces (Neumann conditions), which 
are imposed via the 
\begin{verbatim}
      void
      BoundaryEquation(...)
      {
        ...   
      }
\end{verbatim}

As functionals, we evaluate the pressure in two different 
points of the domain. The observation should be that the 
pressure first starts increasing reaching a maximum and then
starts decaying. This is the so-called Mandel-Creyer effect.
