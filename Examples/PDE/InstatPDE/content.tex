Until now, \dope{} provides various time-stepping schemes
that are based on finite differences. Specifically, the 
user can choose between the 
\begin{itemize}
\item Forward Euler scheme (FE), which is an explicit timestepping scheme. Here, 
one as to take into account that $k\leq ch^2$ where $k$ denotes the timestep size
and $h$ the local mesh cell diameter.
\item Backward Euler scheme (BE), which is an implicit timestepping scheme.
It is strongly A-stable but only from first order and very dissipative. 
The BE-scheme is well suited for stationary numerical examples.  
\item Crank-Nicolson scheme (CN), which is of second order, 
A-stable, has very little dissipation but suffers from case to case from instabilities caused by rough
initial- and/or boundary data. These properties are due to weak stability (it is not \textit{strongly}
A-stable).
\item Shifted (or stabilized) Crank-Nicolson scheme (CN shifted), which is
  also of second order, but provides global stability.
\item Fractional-step-$\theta$ scheme (FS). 
It has second-order accuracy and is strongly A-stable, and therefore
well-suited for computing solutions with rough data. 
\end{itemize}
