\subsubsection{General problem description}
Within this example, we demonstrate how to solve a system of PDEs on a
network\index{PDE on network}.
The example contains the trivial network consisting of the two lines 
\[
(0,50) \qquad (50,100)
\]
and we solve the PDEs 
\begin{align*}
  \partial_x w &= 1& w(0)&=10,\\
  -\partial_x p &= -1& p(100)&=1.
\end{align*}
together with the coupling conditions 
\[
w^-(50) = w^+(50),\qquad p^-(50) = p^+(50).
\]
The functionals evaluate the $L^1$-error between the computed discrete
solutions and the known exact solution for this problem.

\subsubsection{Implementational Details}
In order to realize the coupling across multiple domains we no longer
use a standard \texttt{dealii::DoFHandler} but rather one for each
domain. Internally, the data is stored in a
\texttt{dealii::BlockVector}, thus only a standard
\texttt{dealii::Vector} may be used on each domain, otherwise the
linker fails since, e.g., no implementation of the needed 
\texttt{dealii::BlockVector<BlockVector>} exists.
The first blocks contain each the variables for one of the PDEs on the
given domains. The last block stores the boundary values for each of
the given domains, first all `left' conditions than all `right'
conditions (corresponding to the boundary\_id $0$ and $1$
respectively).

\paragraph{main.cc}
In the main, not much is changing, for many of the already known
classes there is an update given in the \texttt{namespace
  DOpE::Networks} which internally handle the correct selection of
subdomains.
Further, we have to initialize triangulations for each of the domains
and provide a description of the network topology given in the class \texttt{LocalNetwork}.

\paragraph{localpde.h}
In this class, as always we implement the local element and boundary
integrals (this is unchanged), and the same object is used on all
subdomains, hence if different PDEs need to be coupled these need to
be selected, e.g., by setting the \texttt{material\_id} accordingly.
The only difference is that now, we have to access the externally
given boundary values from the other domains (these values are 
now called fluxes). This can be done by
calling \texttt{GetFluxValues} from 
the \texttt{Networks::Network\_FaceDataContainer}.

For the network, there are some new methods to be implemented.

\texttt{BoundaryEquation\_BV} contains the derivative of the
boundary equation with respect to the given flux variables and store
them according to the residual line in which they appear.

\texttt{BoundaryMatrix\_BV} implements the corresponding
matrix. Notice, that it is implicitly assumed that the boundary values
depend linearly on the fluxes!

\texttt{OutflowValues} implements a selection routine to specify which
of the fluxes are `OutflowValues', i.e., those values that are not
considered in the PDE and thus the corresponding flux must be adjusted
in order to match the given outflow.
In addition, this method must set Boolean flags indicating if a value
is to be considered as outflow.

\texttt{OutflowMatrix} the matrix corresponding to the previous
function. Again, it is assumed that the OutflowValues depend linearly
on the solution variable.

\texttt{PipeCouplingResidual} and \texttt{CouplingMatrix} just pass
arguments to the network.

\paragraph{localnetwork.h} This class implements the topology
information needed on the network. In particular:

\texttt{PipeCouplingResidual} needs to evaluate the residual in the 
coupling conditions. These conditions are always \texttt{NPipes}
outflow conditions and an additional \texttt{NPipes} coupling
conditions between the different flux variables, i.e., the boundary
values for $w$ and $p$ and the continuity condition at the point $50$.
It is assumed, again, that these conditions are (affine)-linear.

\texttt{CouplingMatrix} calculates the matrix corresponding to the
coupling conditions.

\texttt{GetFluxSparsityPattern} is needed to calculate the 
sparsity pattern for the coupling ($2\cdot \text{NPipes} \cdot
\text{NComp} \times 2\cdot \text{NPipes} \cdot
\text{NComp}$) matrix. 

