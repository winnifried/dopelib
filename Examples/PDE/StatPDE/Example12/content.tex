\subsubsection{General problem description}
This example shows the use of the Raviart-Thomas (RT) element in \texttt{DOpElib}. The example 
is taken from \texttt{dealii} step-20 and shows how this step can be implemented in \texttt{DOpElib}. The vector valued laplace equation is solved in the mixed formulation.

Most things are identical to all the other programs. However there is a subtle difference in 
\texttt{localpde.h}: Although the element used has 3 components (two for the RT-Element and 1 for the pressure)
the block\underline{ }component vector mapping blocks to components has only 2 entries. 1 for each finite element used!

The second difference is that we have to initialize the mapping explicitly. This is due to the fact that the 
default 
\begin{verbatim}
  DOpEWrapper::Mapping<DIM,DOFHANDLER> mapping(1,false);
\end{verbatim}
is not working with the RT-element and leads to elements on which the divergence is NaN.

An additional feature of this example is that for the solution of the PDE in mixed form we are using the 
Schur complement solver provided in \texttt{dealii} step-20. Hence this example shows how simple it is to 
use self-made linear solvers within the \texttt{DOpElib} framework.