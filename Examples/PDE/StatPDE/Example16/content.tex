\subsubsection{General problem description}
Within this example, we solve the obstacle problem\index{Obstacle problem}
of finding $u \in \mathcal K$ solving 
\begin{equation}\label{ex16:vi}
  (\nabla u, \nabla (\phi-u)) \ge (f,\phi-u) \quad \forall v \in
  \mathcal K
\end{equation}
where
\[
  \mathcal K = \{v \in H^1_0(\Omega)\,|\, v \ge \chi \text{ a.e. in }\Omega\}
\]
on $\Omega = (-1,1)^2$ with
\[
 \chi(x) = \operatorname{dist}(x,\partial \Omega) - 2
 \operatorname{dist}\Bigl(x,\Omega \setminus (-\tfrac{1}{4},\tfrac{1}{4})\Bigr) -\frac{1}{5}
\]
and $f = -5$ as inspired
by~\cite[Example~7.5]{NochettoSiebertVeeser:2003}.
To handle the variational inequality \index{variational inequality},
we introduce a Lagrange multiplier $\lambda$ such that~\eqref{ex16:vi}
is equivalent to finding $u$ and $\lambda$ solving 
\begin{equation}\label{ex16:lagrange}
  \begin{aligned}
    (\nabla u, \nabla \phi) - (f,\phi) - (\lambda,\phi) &= 0 & \forall
    v &\in H^1_0(\Omega),\\
    \lambda & \ge 0,\\
    u-\chi & \ge 0,\\
    (\lambda,u-\chi) &=0.
  \end{aligned}
\end{equation}
Then, we replace the last three inequalities by a complementarity
function, i.e., we notice that for any $c > 0$ 
\[
  x-\max(0,x-cy) = 0 \quad\Leftrightarrow \quad x \ge 0, y \ge 0, xy
  = 0,
\]
and obtain the formulation
\begin{equation}\label{ex16:complementarity}
  \begin{aligned}
    (\nabla u, \nabla \phi) - (f,\phi) - (\lambda,\phi) &= 0 & \forall
    v &\in H^1_0(\Omega),\\
    \lambda - \max(0,\lambda-c(u-\chi)) &= 0.
  \end{aligned}
\end{equation}
For its discretization, we let $\chi_h = I_h \chi$ be the $\mathcal
Q_1$ interpolation of the obstacle and take $u_h \in \mathcal Q_1$.
For the multiplier $\lambda$, we utilize the dual basis of $\mathcal
Q_1$ to define $\mathcal Q_1^*$, i.e., if $\phi^i$ are the nodal basis
functions for $\mathcal Q_1$, then we define the basis $\psi^i$ of
$\mathcal Q_1^*$ from
\[
 (\phi^i,\psi^j) = \delta_{ij} = \begin{cases} 1 & i = j,\\ 0 &
   \text{otherwise.}
 \end{cases}
\] 
This has the advantage, that
the first equation in \eqref{ex16:complementarity} just gets
\[
  (\nabla u, \nabla \phi^i) - (f,\phi^i) - \lambda_i = 0 \quad \forall
    \phi^i \;\text{nodal basis function of $\mathcal Q_1$.}
\]
The second equation, we enforce in the corresponding vertices $x_i$ to
$\phi^i$, only, and get
\[
    \lambda_i - \max(0,\lambda_i-c(u(x_i)-\chi(x_i))) = 0.
\]
 The advantage of this formulation is that we don't need to actually
 evaluate the basis functions $\psi^i$, since we only need the values
 of $\lambda_i$ which we simply store in a $\mathcal Q_1$
 Finite Element at the vertices since then
 \[
   \lambda_i = \lambda(x_i) 
 \]
 and we use a Gauss-Lobatto quadrature to actually evaluate the
 functions at the vertices. The ugly part is that we need to assert
 that we never evaluate $\lambda$ at a non-vertex quadrature point.

 Now, when running over the mesh, we will have each vertex $x_i$ in
 multiple elements, thus counting its contribution more than once.
 To normalize it by the number of times it is counted, we need to
 divide by this number of elements. We can get it from the
 \texttt{ElementDataContainer} using
 \texttt{GetNNeighbourElementsOfVertex}.
 To use it we need to have the function \texttt{HasVertices} in
 \texttt{LocalPDE} to return true, so that the information is generated.
 