\documentclass[a4paper,cleardoubleempty]{scrreprt}
\usepackage{amsmath,amssymb}
\usepackage[pdftex,bookmarks=false,pdfpagelabels=true,plainpages=false]{hyperref}
\usepackage{booktabs}
\usepackage{makeidx}
%\usepackage{sfmath}
\newtheorem{Problem}{Problem}[section]

\newcommand{\todo}[1]{\textbf{\textsc{\textcolor{black}{TODO: #1}}}}


\makeindex

%%%%%%%%%%%%%%%%%%%%%%%%%%%%%%%%%%%%%%%%%%%%%%%%%%
\begin{document}
\title{DOpE - Deal Optimization Environment}
\author{
  Michael Geiger\thanks{University of Heidelberg, 
    {michael.geiger@iwr.uni-heidelberg.de}}, 
  Christian Goll\thanks{University of Heidelberg, 
    {christian.goll@iwr.uni-heidelberg.de}}, 
  Thomas Wick\thanks{University of Heidelberg, 
    {thomas.wick@iwr.uni-heidelberg.de}}, 
  Winnifried Wollner\thanks{University of Hamburg, 
    {winnifried.wollner@math.uni-hamburg.de}}
}



\date{\today}

%%%%%%%%%%%%%%%%%%%%%%%%%%%%%%%%%%%%%%%%%%%%%%%%%%
\maketitle
\cleardoublepage\pagestyle{headings}
\tableofcontents
\cleardoublepage

%%%%%%%%%%%%%%%%%%%%%%%%%%%%%%%%%%%%%%%%%%%%%%%%%%
\chapter{Foreword}
In this report, we describe the \dope{}
\textit{Differential Equations and  Optimization Environment} project. 
Originally, the project was initiated in the year 2009 at the 
University of Heidelberg (Germany) in the numerical analysis
group of Rolf Rannacher. 

The main feature of \dope{} is to give a unified interface to high level 
algorithms such as 
time-stepping methods, nonlinear solvers and optimization routines. 
\dope{} is designed in such a way that the user only needs to write those parts
of the code that are problem dependent while all invariant 
parts of the algorithms
are reusable without any need for further coding.
In particular, the user is enabled to switch between various different 
algorithms without the need to rewrite the problem dependent code, 
though obviously he or she will
have to replace the algorithm object with an other one. 
This replacement can be done by replacing the appropriate object at only
one point in the code.
In addition to the finite element code provided by deal.II --which at present is the only FE-toolkit to which we provide an 
interface-- 
the presented library
\dope{} is user-focused by delivering
prefabricated tools which require adjustments by the user only for parts
connected to his specific problem. 
This is in contrast to deal.II which leaves the implementation of all high-level algorithms to the user.

An innovative feature of \dope{} is to provide a software toolkit to solve forward PDE
problems as well as optimal control problems constrained by PDE. 
\dope{} concentrates on a unified approach for both linear and nonlinear
problems by interpreting every PDE problem as nonlinear and applying a
Newton method to solve it. 
The focus is on the numerical solution of both stationary and nonstationary
problems which come from different application fields, like elasticity and
plasticity, fluid dynamics, and multiphysics problems such as 
fluid-structure interactions.


%%%%%%%%%%%%%%%%%%%%%%%%%%%%%%%%%%%%%%%%%%%%%%%%%%
At the present stage the following features are supported by the library
\begin{itemize}
\item Solution of stationary and nonstationary PDEs in 1d, 2d, and 3d.
\item Various time stepping schemes (based on finite differences), 
  such as forward Euler, backward Euler,
  Crank-Nicolson, shifted Crank-Nicolson, and Fractional-Step-$\Theta$ scheme.
\item All finite elements of from deal.II including hp-support.
\item Self-written line search and trust region newton algorithms for the 
   solution of optimization problems with PDEs \cite{NoWr00}
\item Interface to SNOPT for the solution of optimization problems with PDEs and
  additional other constraints.
\item Several examples showing how to solve various kinds of optimization problems
  involving stationary PDE constraints.
\item Mesh adaptation and goal-oriented error estimation with
the dual-weighted residual (DWR) method.
\item Different spatial triangulations for control and state variables.
\item Several examples showing the solution of several PDEs including
   Poisson, Navier-Stokes, plasticity, fluid-structure interaction problems,
a coupled Biot-Lam\'e-Navier system, and finally from financial mathematics
the Black-Scholes equations.
\end{itemize}


The rest of this document is structured as follows: We start with an introduction in
Chapter~\ref{chap:intro} where you will learn what is needed to run \dope{}. 
Further you will learn what problems we can solve and how all the different classes 
work together for this purpose. This should help you figure out what the different classes
do if you are in need of writing your own algorithm.

Then assuming that you can work to your satisfaction with the algorithms already implemented
we will show you how to create your own running example in Chapter~\ref{chap:howtoex}.
This will be followed by a detailed description of all examples already shipped with 
the library. You can find the examples for the solution of PDEs in Chapter~\ref{PDE}
and those for the solution of optimization problems with PDEs in Chapter~\ref{OPT}.

These notes conclude with a section that explains how we do automated testing of the 
implementation in Chapter~\ref{chap:test}. This chapter will be of interest only if you 
are trying to implement some new features to the library so that you can check that 
the new code did not break anything.

%\paragraph{Contributors \& Thanks:}
\paragraph{Thanks:}
The \dope{} project is 
mainly based on the \textit{deal.II} finite element library which has been developed
 initially by W. Bangerth, R. Hartmann, and G. Kanschat \cite{deal} and 
now maintained by \cite{deal2}. Special thanks go to them!

In addition to deal.II, the authors 
gratefully acknowledge their past experience as well as discussions with 
the authors of  
the software toolkits \texttt{Gascoigne} and \texttt{RoDoBo}
\cite{Gascoigne} and \cite{rodobo-web}, 
from which some of the ideas to modularize the algorithms have arisen.
Similar thanks go to the developers of \texttt{Ipopt},~\cite{ipopt}.

Last, but not least, we would like to express our gratitude 
to all contributors listed in Section~\ref{sec:contrib}
for their respective contribution to the library.

Special thanks go to Christian Goll who has been a great help in 
maintaining and developing this library from 2009 until 2015. 
 

%{We wish to express our gratitude to those people that have contributed to the library or
%  have been providing open software to which we have provided interface.
%
%
%These are the developers of the software toolkits \texttt{Gascoigne} and \texttt{RoDoBo}
%\cite{Gascoigne} and \cite{rodobo-web}, as their software as well as the discussions 
%have shaped some of the initial design ideas for this library. 
%
%Further, we would like the thank the developers of the \texttt{deal.II} finite element library 
%\cite{deal,DealIIReference} for providing the finite elements to which we have interfaced. 
%









\cleardoublepage

%%%%%%%%%%%%%%%%%%%%%%%%%%%%%%%%%%%%%%%%%%%%%%%%%%
\chapter{Example Handling, Creating new Examples}
To implement new examples or to use existing examples 
from the library for own research, the user 
can simply copy an existing example. In this 
new example, own code and changes can be compiled.

Adding a new regression test to the repository 
needs some concentration of the user. In the 
following, we explain how to do this and what 
\textit{must} be taken into account:
\begin{verbatim}
cp -r Example1 ExampleNew
cd ExampleNew
rm -r CVS
cd Test
rm -r CVS
\end{verbatim}
It is important to remove the repository information,
which is stored in the directory CVS.

Then, we have to consider the following step. In order to 
run the regression tests, we start the executable file
of the present example. This executable comes from the 
Makefile. For example in the Makefile of 
Example \ref{PDE_Stat_Stokes}, 
\begin{verbatim}
target   = $(BINDIR)/DOpE-PDE-StatPDE-Example1-$
           (dope_dimension)d-$(deal_II_dimension)d
\end{verbatim}
and concretely:
\begin{verbatim}
DOpE-PDE-StatPDE-Example1-2d-2d
\end{verbatim}
The aforementioned expression is the name of 
the executable that is required to run the 
regression test. Hence, in 
\begin{verbatim}
PDE/StatPDE/Example1/Test> emacs test.sh
\end{verbatim}
you find two times the line
\begin{verbatim}
echo "Running Program 
     ../../../../../bin/DOpE-PDE-StatPDE-Example1-2d-2d test.prm"
     (../../../../../bin/DOpE-PDE-StatPDE-Example1-2d-2d test.prm 2>&1) 
     > /dev/null
\end{verbatim} 
It is really important when copying an existing example,
to change this information. 
First, we have to change the target in the Makefile
\begin{verbatim}
target   = $(BINDIR)/DOpE-PDE-StatPDE-ExampleNew-$
           (dope_dimension)d-$(deal_II_dimension)d
\end{verbatim}
Second, we have to replace 
Example1 by ExampleNew such that 
\begin{verbatim}
echo "Running Program 
     ../../../../../bin/DOpE-PDE-StatPDE-ExampleNew-2d-2d test.prm"
     (../../../../../bin/DOpE-PDE-StatPDE-ExampleNew-2d-2d test.prm 2>&1) 
     > /dev/null
\end{verbatim} 
Please, also change the dimension parameters 
\begin{verbatim}
-2d-2d
\end{verbatim} 
Now, the user is prepared to change any information in the 
\begin{verbatim}
main.cc, localpde.h, functionals.h, localfunctional.h, etc
\end{verbatim} 



%%%%%%%%%%%%%%%%%%%%%%%%%%%%%%%%%%%%%%%%%%%%%%%%%%
\chapter{Examples for PDE Solution}
\label{PDE}
%%%%%%%%%%%%%%%%%%%%%%%%%%%%%%%%%%%%%%%%%%%%%%%%%%
\section{Stationary PDEs} 
\label{PDE_Stat}
%%%%%%%%%%%%%%%%%%%%%%%%%%%%%%%%%%%%%%%%%%%%%%%%%%
\subsection{Stationary Stokes Equations} 
\label{PDE_Stat_Stokes}
todo. short: this example shows how to use the constraintsmaker class by enforcing periodic boundary conditions in 2d.

\clearpage
%%%%%%%%%%%%%%%%%%%%%%%%%%%%%%%%%%%%%%%%%%%%%%%%%%
\subsection{Stationary FSI with INH Material} 
\label{PDE_Stat_FSI_INH}
todo. short: this example shows how to use the constraintsmaker class by enforcing periodic boundary conditions in 2d.

\clearpage
%%%%%%%%%%%%%%%%%%%%%%%%%%%%%%%%%%%%%%%%%%%%%%%%%%
\subsection{Stationary FSI with STVK Material} 
\label{PDE_Stat_FSI_STVK}
todo. short: this example shows how to use the constraintsmaker class by enforcing periodic boundary conditions in 2d.

\clearpage
%%%%%%%%%%%%%%%%%%%%%%%%%%%%%%%%%%%%%%%%%%%%%%%%%%
\subsection{Stationary Elasticity Benchmark} 
\label{PDE_Stat_Elasticity}
todo. short: this example shows how to use the constraintsmaker class by enforcing periodic boundary conditions in 2d.

\clearpage
%%%%%%%%%%%%%%%%%%%%%%%%%%%%%%%%%%%%%%%%%%%%%%%%%%
\subsection{Stationary Plasticity Benchmark} 
\label{PDE_Stat_Plasticity}
todo. short: this example shows how to use the constraintsmaker class by enforcing periodic boundary conditions in 2d.

\clearpage
%%%%%%%%%%%%%%%%%%%%%%%%%%%%%%%%%%%%%%%%%%%%%%%%%%
\subsection{Stationary Stokes Equations with periodic BC} 
\label{PDE_Stat_Periodic_Stokes}
todo. short: this example shows how to use the constraintsmaker class by enforcing periodic boundary conditions in 2d.

\clearpage
%%%%%%%%%%%%%%%%%%%%%%%%%%%%%%%%%%%%%%%%%%%%%%%%%%
\subsection{Laplace Equation in 2D} 
\label{PDE_Stat_Laplace_2D}
todo. short: this example shows how to use the constraintsmaker class by enforcing periodic boundary conditions in 2d.

\clearpage
%%%%%%%%%%%%%%%%%%%%%%%%%%%%%%%%%%%%%%%%%%%%%%%%%%
\subsection{Laplace Equation in 3D}
\label{PDE_Stat_Laplace_3D}
todo. short: this example shows how to use the constraintsmaker class by enforcing periodic boundary conditions in 2d.

\clearpage
%%%%%%%%%%%%%%%%%%%%%%%%%%%%%%%%%%%%%%%%%%%%%%%%%%
\section{Nonstationary PDEs}
\label{PDE_Instat}
todo. short: this example shows how to use the constraintsmaker class by enforcing periodic boundary conditions in 2d.

%%%%%%%%%%%%%%%%%%%%%%%%%%%%%%%%%%%%%%%%%%%%%%%%%%
\subsection{Nonstationary Stokes Equations}
\label{PDE_Instat_Stokes}
todo. short: this example shows how to use the constraintsmaker class by enforcing periodic boundary conditions in 2d.

\clearpage
%%%%%%%%%%%%%%%%%%%%%%%%%%%%%%%%%%%%%%%%%%%%%%%%%%
\subsection{Nonstationary FSI Problem}
\label{PDE_Instat_FSI}
todo. short: this example shows how to use the constraintsmaker class by enforcing periodic boundary conditions in 2d.

\clearpage
%%%%%%%%%%%%%%%%%%%%%%%%%%%%%%%%%%%%%%%%%%%%%%%%%%
\subsection{Black-Scholes Equation}
\label{PDE_Instat_Black_Scholes}
todo. short: this example shows how to use the constraintsmaker class by enforcing periodic boundary conditions in 2d.

\clearpage
%%%%%%%%%%%%%%%%%%%%%%%%%%%%%%%%%%%%%%%%%%%%%%%%%%
\subsection{Heat Equation in 1D}
\label{PDE_Instat_Heat_1D}
todo. short: this example shows how to use the constraintsmaker class by enforcing periodic boundary conditions in 2d.

\clearpage
%%%%%%%%%%%%%%%%%%%%%%%%%%%%%%%%%%%%%%%%%%%%%%%%%%
\subsection{Heat Equation in 2D with nonlinearity}
\label{PDE_Instat_Heat_2D}
todo. short: this example shows how to use the constraintsmaker class by enforcing periodic boundary conditions in 2d.

\cleardoublepage
%%%%%%%%%%%%%%%%%%%%%%%%%%%%%%%%%%%%%%%%%%%%%%%%%%
\chapter{Examples with Optimization}
\label{OPT}
%%%%%%%%%%%%%%%%%%%%%%%%%%%%%%%%%%%%%%%%%%%%%%%%%%
\section{Subject to a Stationary PDE}
\label{OPT_Stat}
%%%%%%%%%%%%%%%%%%%%%%%%%%%%%%%%%%%%%%%%%%%%%%%%%%
\subsection{Distributed control with a linear elliptic PDE}
\label{OPT_Stat_Distrib_Lin_Ellipt}
todo. short: this example shows how to use the constraintsmaker class by enforcing periodic boundary conditions in 2d.

\clearpage
%%%%%%%%%%%%%%%%%%%%%%%%%%%%%%%%%%%%%%%%%%%%%%%%%%
\subsection{Parameter control with a linear elliptic PDE}
\label{OPT_Stat_Param_Lin_Ellipt}
todo. short: this example shows how to use the constraintsmaker class by enforcing periodic boundary conditions in 2d.

\clearpage
%%%%%%%%%%%%%%%%%%%%%%%%%%%%%%%%%%%%%%%%%%%%%%%%%%
\subsection{Parameter control with a nonlinear PDE from fluid dynamcis}
\label{OPT_Stat_Param_Nonlin_Fluid}
todo. short: this example shows how to use the constraintsmaker class by enforcing periodic boundary conditions in 2d.

\clearpage
%%%%%%%%%%%%%%%%%%%%%%%%%%%%%%%%%%%%%%%%%%%%%%%%%%
\subsection{Control in the dirichlet boundary values}
\label{OPT_Stat_Dirichlet_Boundary}
todo. short: this example shows how to use the constraintsmaker class by enforcing periodic boundary conditions in 2d.

\clearpage
%%%%%%%%%%%%%%%%%%%%%%%%%%%%%%%%%%%%%%%%%%%%%%%%%%
\subsection{Distributed Control with Different Meshes for Control and State}
\label{OPT_Stat_Distributed_MultiMesh}
todo. short: this example shows how to use the constraintsmaker class by enforcing periodic boundary conditions in 2d.

\clearpage
%%%%%%%%%%%%%%%%%%%%%%%%%%%%%%%%%%%%%%%%%%%%%%%%%%
\subsection{Compliance Minimization of a variable Thickness MBB-Beam}
\label{OPT_Stat_MBB-Beam}
todo. short: this example shows how to use the constraintsmaker class by enforcing periodic boundary conditions in 2d.

\clearpage
%%%%%%%%%%%%%%%%%%%%%%%%%%%%%%%%%%%%%%%%%%%%%%%%%%
\subsection{Distributed control with a linear elliptic PDE using SNOPT}
\label{OPT_Stat_Box_controlconstraints_SNOPT}
todo. short: this example shows how to use the constraintsmaker class by enforcing periodic boundary conditions in 2d.

\clearpage
%%%%%%%%%%%%%%%%%%%%%%%%%%%%%%%%%%%%%%%%%%%%%%%%%%
\subsection{Topology optimization of an MBB-Beam using SNOPT}
\label{OPT_Stat_TopOpt_MBB_SNOPT}
todo. short: this example shows how to use the constraintsmaker class by enforcing periodic boundary conditions in 2d.

\clearpage
%%%%%%%%%%%%%%%%%%%%%%%%%%%%%%%%%%%%%%%%%%%%%%%%%%
\subsection{Parameter control with a non-linear PDE from FSI dynamics}
\label{OPT_Stat_Param_Nonlin_FSI}
%todo. short: this example shows how to use the constraintsmaker class by enforcing periodic boundary conditions in 2d.

\clearpage


%%%%%%%%%%%%%%%%%%%%%%%%%%%%%%%%%%%%%%%%%%%%%%%%%%
\chapter{Testing examples}
The DOpE testsuite consists of regression tests. They are run to compare 
the output to previous outputs. This is useful (necessary) after 
changing programming code anywhere in the library. If a test
succeeds, everything is fine in the library. If not, you should not
check in your code into DOpE. Please make sure what is going wrong and WHY!

Every command is computed via a Makefile. In the basic example 
directory

%%%%%%%%%%%%%%%%%%%%%%%%%%%%%%%%%%%%%%%%%%%%%%%%%%
\section{Where can I find the tests}
In each example directory you find a subdirectory `Test'. Herein, you find 
the parameter files for meshes (*.inp) and a param file (test.prm).
Moreover, the executable is denoted by `test.sh'. Please make sure, that 
the 
\begin{verbatim}
set never_write_list
\end{verbatim}
contains every possible output
\begin{verbatim}
Gradient;Hessian;Tangent;Residual;Update;Control;State
\end{verbatim}
That means, no solution files are written to the output.
Recall, that we are just interested in terminal output that 
is of course sufficient to verify the things.

Hence, the results directory should be empty
\begin{verbatim}
set results_dir   = ./
\end{verbatim}
The rest in the param file must be identically the same as 
in the dope.prm file in the parent directory. 

%%%%%%%%%%%%%%%%%%%%%%%%%%%%%%%%%%%%%%%%%%%%%%%%%%
\section{How to start testing?}
You start testing by typing 
\begin{verbatim}
> ./test.sh Store
\end{verbatim}
in the terminal.

After the run, you have to call 
\begin{verbatim}
> ./test.sh Test
\end{verbatim}
to compare your stored output. Of course, there should be no differences.

The useful point is now the following. After implemention of 
new pieces of code in the DOpE library or in the examples, you can
run 
\begin{verbatim}
> ./test.sh Test
\end{verbatim}
Hereby, you compare your `new' output with the previous stored output.

Attention: After changes you should NOT run again
\begin{verbatim}
> ./test.sh Store
\end{verbatim}
In that case, you overwrite your previous output.
\cleardoublepage





%%%%%%%%%%%%%%%%%%%%%%%%%%%%%%%%%%%%%%%%%%%%%%%%%%
\addcontentsline{toc}{chapter}{Index}
\printindex

%%%%%%%%%%%%%%%%%%%%%%%%%%%%%%%%%%%%%%%%%%%%%%%%%%
\end{document}
